\chapter{Particle Physics}
\label{chap:particlephysics}

This chapter will be an introduction to particle physics.
The purpose is to introduce the reader to the concepts of subatomic particles,
and to introduce the names of the specific particles.

\section{The Standard Model}
\label{sec:standardmodel}

\subsection{Description}
\label{ssec:SMdescription}

The Standard Model (SM) is...
What are all the particles of the Standard Model?
Give particular emphasis to the top quark

\subsection{Successes}
\label{ssec:SMsuccesses}

Correctly predicted pretty much every physics result in the last... what, 30 years?

\subsection{Shortcomings}
\label{ssec:SMshortcomings}

Gravity / hierarchy problem, Higgs mass surprise,
Dark Matter, (Dark Energy?)

\section{Supersymmetry}
\label{sec:susy}

Also known as SUSY

\subsection{Motivation}
\label{ssec:susymotivation}

It's possible this part should be split between the preceding and
subsequent sections. Not sure.

\subsection{Principles}
\label{ssec:susyprinciples}

Explain about sparticles, the differences from particles,
the naming conventions, and how they fix some of the problems
of the Standard Model

\subsection{Sparticles}
\label{ssec:susysparticles}

I think this is somewhat redundant. It might wind up fitting more
naturally into the previous section. Or who knows, maybe this entire
section will wind up having NO subsections at all. IDK.

\section{Top quarks as a window to BSM physics}
\label{sec:topquarks}

Explain why the heavy mass of the top quark makes it a natural link
to physics beyond the standard model.