%
%
% UCSD Doctoral Dissertation Template
% -----------------------------------
% http://ucsd-thesis.googlecode.com
%
%


%% REQUIRED FIELDS -- Replace with the values appropriate to you

% No symbols, formulas, superscripts, or Greek letters are allowed
% in your title.
\title{Two Tests of New Physics Using Top Quarks at the CMS Experiment}
% Two analyses involving top quarks and new physics at CMS
% Top Quarks and Stop Squarks at the CMS Experiment
% Top Quarks and New Physics at the CMS Experiment
% Two Searches for New Physics using Top Quarks at the CMS Experiment

\author{Daniel Stuart Klein}
\degreeyear{\the\year}

% Master's Degree theses will NOT be formatted properly with this file.
\degreetitle{Doctor of Philosophy}

\field{Physics}
%\specialization{Blah}  % If you have a specialization, add it here

\chair{Professor Frank W\"{u}rthwein}
% Uncomment the next line iff you have a Co-Chair
\cochair{Professor Avraham Yagil}
%
% Or, uncomment the next line iff you have two equal Co-Chairs.
%\cochairs{Professor Chair Masterish}{Professor Chair Masterish}

%  The rest of the committee members  must be alphabetized by last name.
\othermembers{
Professor Rommie Amaro\\
Professor Pamela Cosman\\
Professor Benjam\'{i}n Grinstein\\
}
\numberofmembers{5} % |chair| + |cochair| + |othermembers|


%% START THE FRONTMATTER
%
\begin{frontmatter}

%% TITLE PAGES
%
%  This command generates the title, copyright, and signature pages.
%
\makefrontmatter

%% DEDICATION
%
%  You have three choices here:
%    1. Use the ``dedication'' environment.
%       Put in the text you want, and everything will be formated for
%       you. You'll get a perfectly respectable dedication page.
%
%
%    2. Use the ``mydedication'' environment.  If you don't like the
%       formatting of option 1, use this environment and format things
%       however you wish.
%
%    3. If you don't want a dedication, it's not required.
%
%
\begin{dedication}
  This dissertation is dedicated\\
  to the memory of my grandmother, Dr. Isabelle Rapin (1927-2017).

  To everyone else, a titan of pediatric neurology.

  To me, one of my biggest cheerleaders.
\end{dedication}


% \begin{mydedication} % You are responsible for formatting here.
%   \vspace{1in}
%   \begin{flushleft}
% 	To me.
%   \end{flushleft}
%
%   \vspace{2in}
%   \begin{center}
% 	And you.
%   \end{center}
%
%   \vspace{2in}
%   \begin{flushright}
% 	Which equals us.
%   \end{flushright}
% \end{mydedication}



%% EPIGRAPH
%
%  The same choices that applied to the dedication apply here.
%
\begin{epigraph} % The style file will position the text for you.
  \emph{Now you see why your father and I\\
    like to call it 'gradual school'.}\\
  ---Mom
\end{epigraph}

% \begin{myepigraph} % You position the text yourself.
%   \vfil
%   \begin{center}
%     {\bf Think! It ain't illegal yet.}
%
% 	\emph{---George Clinton}
%   \end{center}
% \end{myepigraph}


%% SETUP THE TABLE OF CONTENTS
%
\tableofcontents
\listoffigures  % Comment if you don't have any figures
\listoftables   % Comment if you don't have any tables



%% ACKNOWLEDGEMENTS
%
%  While technically optional, you probably have someone to thank.
%  Also, a paragraph acknowledging all coauthors and publishers (if
%  you have any) is required in the acknowledgements page and as the
%  last paragraph of text at the end of each respective chapter. See
%  the OGS Formatting Manual for more information.
%
\begin{acknowledgements}
My acknowledgements will eventually go here. %%%%%%%%%%%%%%%%%%%%%%%%%%%%%%%%%%%%%%%%%%%%%%%
\end{acknowledgements}


%% VITA
%
%  A brief vita is required in a doctoral thesis. See the OGS
%  Formatting Manual for more information.
%
\begin{vitapage}
\begin{vita}
  \item[2011] B.~A. in Physics and Mathematics, Cornell University, Ithaca
  \item[2013] M.~S. in Physics, University of California, San Diego
  \item[2018] Ph.~D. in Physics, University of California, San Diego
\end{vita}
\begin{publications}
  \item ``Search for top squark pair production in pp collisions at $\sqrt{s}$ = 13 TeV using single lepton events'', CMS Collaboration, \emph{JHEP} \textbf{10}, 019 (2017) %https://link.springer.com/article/10.1007%2FJHEP10%282017%29019
  \item ``Searches for pair production for third-generation squarks in $\sqrt{s}$ = 13 TeV pp collisions'', CMS Collaboration, \emph{Eur. Phys. Jour.} \textbf{C77}, 327 (2017) %https://link.springer.com/article/10.1140%2Fepjc%2Fs10052-017-4853-2
  \item ``Measurements of $t\bar{t}$ charge asymmetry using dilepton final states in pp collisions at $\sqrt{s}$ = 8 TeV'', CMS Collaboration, \emph{Phys. Lett.} \textbf{B760}, 365 (2016) %http://www.sciencedirect.com/science/article/pii/S0370269316303471
  \item ``Measurements of $t\bar{t}$ spin correlations and top quark polarization using dilepton final states in pp collisions at $\sqrt{s}$ = 8 TeV'', CMS Collaboration, \emph{Phys. Rev.} \textbf{D93}, 052007 (2016) %http://journals.aps.org/prd/abstract/10.1103/PhysRevD.93.052007
  \item ``Measurements of the $t\bar{t}$ charge asymmetry using the dilepton decay channel in pp collisions at $\sqrt{s}$ = 7 TeV'', CMS Collaboration, \emph{JHEP} \textbf{04}, 191 (2014) %http://link.springer.com/article/10.1007%2FJHEP04%282014%29191
\end{publications}
\end{vitapage}


%% ABSTRACT
%
%  Doctoral dissertation abstracts should not exceed 350 words.
%   The abstract may continue to a second page if necessary.
%
\begin{abstract}
  Two related analyses of data from the CMS Experiment are presented. The
  first is performed using 19.5 inverse femtobarns of proton-proton
  collision data from CMS Run I. In this analysis, six different asymmetry
  variables are measured in events with top-antitop quark pairs
  decaying to final states with two leptons. Unfolding techniques are
  used to extrapolate these measurements to parton level. No deviations
  from the predictions of the Standard Model are found, implying the
  absence of any influence from physics beyond the Standard Model. The
  second analysis is performed using 35.9 inverse femtobarns of
  proton-proton collision data collected during CMS Run II. In
  this analysis, a search is performed for evidence of stop squark pair
  production and decay to single-lepton final states. Several
  backgrounds, including dileptonic top-pair production, are estimated
  using control regions in the data. No excess above the Standard
  Model backgrounds is found, and exclusion limits are placed on three
  models of stop squark pair production.
\end{abstract}


\end{frontmatter}
