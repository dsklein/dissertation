\chapter{Top Asymmetry Measurements}
\label{chap:afb}

This chapter will describe the measurements of several asymmetries in
events where a top-antitop ($\ttbar$) pair decays to a two-lepton
final state ($2\ell$). This work was performed during Run I of the LHC and the
CMS detector, at both 7 TeV and 8 TeV center-of-mass energies, though
I will focus on the 8 TeV results, and in particular, my work on the
unfolding technique. The 8 TeV analysis led to two publications, which
are Refs. \cite{chargeasym} and \cite{spincorrpol}.

% Question: Should I use the past tense of present tense when
% describing this analysis?
% I guess the same question is also applicable to the stop analysis.

\section{Motivation}
\label{sec:afbmotivation}

As mentioned in Section \ref{ssec:SMdescription}, the top quark is the
heaviest known elementary particle, with a mass of $\approx 173$ GeV
\cite{pdg}. This large mass gives the top quark a number of properties
that make it uniquely interesting to study.

In particle physics, a particle's mass is inversely proportional to
its lifetime - how long it tends to last before decaying. % Do I need a citation for this?
Because the top quark is so heavy, it has an extremely short lifetime
of $\mathcal{O} 10^{-25}$ seconds \cite{pdg}.
Because this lifetime is so short, there is insufficient time for the
top quark to hadronize, or for spin decorrelation to occur. % Does this need a separate reference?
Thus, unlike any other quark, the polarization and spin correlation
of the top quark are preserved in its decay products. So by studying
the decays of top quarks, we can learn about the properties of the
bare quark itself.

Additionally, the top quark's large mass places it on the edge of
unexplored territory.
Because the likelihood of a particular decay is proportional
to the mass difference between the particles involved, there are many
models of new physics that predict new particles with a strong
coupling to the top quark. % Probably gonna need to cite or rewrite that claim...
These couplings would naturally cause some of the top quark's
properties to deviate from the value predicted by the SM. Thus, it is
important to make precise measurements of properties such as charge
asymmetry to tease out any links to possible new physics.

\section{Previous Measurements}
\label{sec:afbtevatron}

The Tevatron collider at Fermilab has made several previous measurements of
top charge asymmetry (CITATIONS), and of top spin % Pull in all these citations
correlation and polarization (CITATIONS). The
early charge asymmetry measurements were particularly interesting, because
they revealed a $2\sigma$ deviation from the SM expectation (though
later results, such as Reference (CITATION), showed a
reduced discrepancy). During the LHC Run I, it was
natural to want to confirm the Tevatron measurements, and to extend
them if possible.

The Tevatron collided beams of protons against beams of
antiprotons. When such $p\bar{p}$ collisions produce
$\ttbar$ pairs, we naturally expect the top will travel more along the direction
of the proton beam, and the antitop will travel
more along the direction of the antiproton beam, so that the
difference in their rapidities is positive ($y_t - y_{\bar{t}} > 0$).
This case was called ``forward'', and the reverse was called
``backward''. Thus, given a number of $\ttbar$
events from the Tevatron, we can define a forward-backward
asymmetry by subtracting the number of backward events from the number
of forward ones:

\begin{equation}
A_{FB} = \asymmetry{\Delta y}{0}
\end{equation}

However, the LHC is a proton-proton collider, and as such, there is no
naturally favored direction for the top or antitop to travel. As the
next section will explain, we must define ``forward'' and ``backward''
events somewhat differently.

\section{Asymmetry Variables}
\label{sec:afbvariables}

In this analysis, we evaluated six different asymmetries in $\ttdilep$
events, including two charge asymmetries, one
polarization, and three measures of spin correlation. If any of these
variables deviate from their expected SM values, it could potentially
be a sign of new physics.

\subsection{Definitions}
\label{ssec:afbvariables}

% Top charge asymmetry %%%%%%%%%%%
As at the Tevatron, the top charge asymmetry was measured at CMS using
the rapidity of the top quarks. However, absent a natural direction
for the top and antitop to travel, we categorized events as
``forward'' and ``backward'' based on which quark had a higher
absolute value of rapidity. Thus we define our \textbf{top charge
asymmetry} as:
\begin{equation}
A_C = \asymmetry{|y_t|}{|y_{\bar{t}}|}
\end{equation}

% Leptonic charge asymmetry %%%%%%%%
In addition to the top charge asymmetry, we define a similar
\textbf{leptonic charge asymmetry} to be:
\begin{equation}
A_C^{lep} = \asymmetry{|\eta_{\ell^+}|}{|\eta_{\ell^-}|}
\end{equation}
Though similar to the top charge asymmetry, this variable
is based purely on the pseudorapidity of the leptons produced in the
decay. We may use the leptons as a proxy for the tops because we
expect their direction to be correlated with the directions of their
parent tops. One advantage of using only the leptons is that it obviates the
need to reconstruct the $\ttbar$ system, as described in Sec.
\ref{sec:afbselections}, thereby allowing us to include more events in
our calculation of this variable. However, this variable also has some
dependence on the polarization, so it is not fully correlated with $A_C$.

% INTERLUDE: Helicity angle %%%%%%%%%%%
Before proceding further, we must define the \emph{helicity angle}, which
will be used in the next two asymmetry variables.
The helicity angle $\theta^*_{\ell}$ is defined as the angle
a charged lepton makes in the rest frame of its parent top,
relative to the parent top's direction in the $\ttbar$ center-of-mass
(CM) reference frame. This angle may be defined separately for the
positive and negative charged lepton in an event.

% Top polarization %%%%%%%%%%%%%%
The \textbf{top polarization} is an asymmetry based on the helicity %% Note: revisit this later, and decide whether to present +/- separately or together
angles of the charged leptons, and is given by:
\begin{equation}
A_{P\pm} = \asymmetry{\cos(\theta^*_{\ell^\pm})}{0}
\end{equation}
We measure it for the positively and
negatively charged leptons separately. A different, commonly used
polarization variable, $P$, can be calculated as $P = 2 A_P = 2(
A_{P+} + A_{P-})$, if we assume CP-invariance.

% Top spin correlation %%%%%%%%%
Our \textbf{top spin correlation} asymmetry, which also depends on the
lepton helicity angles, is defined as:
\begin{equation}
A_{c1c2} = \asymmetry{\cos(\theta^*_{\ell^+}) \times \cos(\theta^*_{\ell^-})}{0}
\end{equation}
From this variable, we may also obtain the $C$ spin correlation
coefficient, which is given by $C = -4 \times A_{c1c2}$ \cite{spincorrpolcoeff}.

% Lepton azimuthal asymmetry %%%%%
We also obtain an indirect measurement of the spin correlation using
the \textbf{lepton azimuthal asymmetry}. This variable depends on the
azimuthal angle between the two leptons, and is defined as:
\begin{equation}
A_{\Delta\phi} = \asymmetry{\Delta\phi_{\ell^+\ell^-}}{\pi/2}
\end{equation}
As with the leptonic charge asymmetry, this variable also relies
solely on the charged leptons, and does not require reconstruction of
the $\ttbar$ system, permitting us to use more events than are
available for the top spin correlation.

% Lepton opening angle %%%%%%%%%
Finally, we define our \textbf{lepton opening angle} asymmetry as:
\begin{equation}
A_{\cos\phi} = \asymmetry{\cos(\theta_{\ell\ell})}{0}
\end{equation}
This variable is based on the opening angle between the two leptons in
their respective parent top reference frames. From this measured
asymmetry, we can calculate the $D$ spin correlation coefficient to be
$D = -2 \times A_{\cos\phi}$ \cite{spincorrpolcoeff}.

\subsection{Differential Measurements}
\label{ssec:afbvarsdifferential}

In addition to measuring the above six asymmetries in one dimension
(\emph{inclusively}), we also measured them in two dimensions
(\emph{differentially}), against other physical variables.
We chose to make differential measurements because the CDF charge asymmetry
discrepancy was more pronounced at $\ttbar$ invariant masses above 450 % Cite CDF charge asymmetry at Mtt>450
GeV than below it. If similar behavior occurs at CMS, in any variable,
we would like to observe it. We measured all six of our asymmetry
variables differentially with respect to the $\ttbar$ invariant mass
($m_{\ttbar}$), rapidity ($y_{\ttbar}$), and transverse momentum ($p_T^{\ttbar}$).

\section{Datasets and Triggers}
\label{sec:afbdatatrig}

Explain which CMS datasets were used.
Explain which CMS Monte Carlo datasets were used.
Explain which triggers were used.

\section{Object and Event Selection}
\label{sec:afbselections}

Give definitions of electrons, muons, jets, MET, etc. used in the analysis.
List how the different objects are used to select events.
Maybe also talk about scale factors and other reweighting here?

Will probably also want to talk about reconstructing the ttbar system
using the Rochester magic.

\section{Comparison Between Data and Simulation}
\label{sec:afbdatamccompare}

Insert plots comparing data and Monte Carlo, to show that our
simulations reflect distributions in data of certain key variables.

\section{Background Estimation}
\label{sec:afbbackground}

Describe the control regions we used, and what each one is suited for.
Talk about how we varied the target MC component to optimize the total
yield in each CR.
Don't forget to mention how high our S/B ratio is.

\section{Unfolding}
\label{sec:afbunfolding}

\subsection{Background}
\label{ssec:afbunfoldingbkg}

Describe the unfolding technique.

\subsection{One-Dimensional Unfolding}
\label{ssec:afbunfolding1d}

Describe how we used 1D unfolding to measure the asymmetry variables.
Talk about binning choices, and all those other optimizations.

\subsection{Two-Dimensional Unfolding}
\label{ssec:afbunfolding2d}

Describe how we used 2D unfolding to measure differential asymmetries.
Talk about binning choices and other optimizations.

\subsection{Validation}
\label{ssec:afbunfoldingtests}

Talk about the linearity and pull tests, and how they quantify any bias
introduced by regularization.
Give the values from those tests.

\section{Systematic Uncertainties}
\label{sec:afbsystematics}

Describe all the systematic uncertainties we calculated.

\section{Results and Interpretation}
\label{sec:afbresults}

\subsection{One-Dimensional Results}
\label{ssec:afbresults1d}

Give the 1D results

\subsection{Two-Dimensional Results}
\label{ssec:afbresults2d}

Give the 2D results

\subsection{BSM Interpretation}
\label{ssec:afbresultsbsm}

Not sure if this section should be included or not. If I keep it:
Talk about the search for chromo-magnetic dipole moments.

\subsection{Acknowledgements}
Mention work of CMS collaboration, maybe specific people who worked on
the analysis with me, and definitely Bernreuther and Si
