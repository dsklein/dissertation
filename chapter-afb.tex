\chapter{Top Asymmetry Measurements}
\label{chap:afb}

\section{Motivation}
\label{sec:afbmotivation}

The top quark is the heaviest elementary particle.
This gives it the unique property that it decays before it can hadronize.
This means that its decay products preserve some of its properties.
We are interested in measuring these properties for various reasons.

\section{Previous Measurements}
\label{sec:afbtevatron}

The Tevatron did this analysis previously.
The Tevatron was a proton-antiproton collider.
This means there's a natural way to define the forward and
backward directions.
The LHC is a proton-proton collider, so the directions are symmetric.
So we have to define forward and backward somewhat differently.

\section{Asymmetry Variables}
\label{sec:afbvariables}

Here I'll explain the various variables measured in the AFB analysis.

\section{Datasets and Triggers}
\label{sec:afbdatatrig}

Explain which CMS datasets were used.
Explain which CMS Monte Carlo datasets were used.
Explain which triggers were used.

\section{Object and Event Selection}
\label{sec:afbselections}

Give definitions of electrons, muons, jets, MET, etc. used in the analysis.
List how the different objects are used to select events.
Maybe also talk about scale factors and other reweighting here?

\section{Comparison Between Data and Simulation}
\label{sec:afbdatamccompare}

Insert plots comparing data and Monte Carlo, to show that our
simulations reflect distributions in data of certain key variables.

\section{Background Estimation}
\label{sec:afbbackground}

Describe the control regions we used, and what each one is suited for.
Talk about how we varied the target MC component to optimize the total
yield in each CR.
Don't forget to mention how high our S/B ratio is.

\section{Unfolding}
\label{sec:afbunfolding}

\subsection{Background}
\label{ssec:afbunfoldingbkg}

Describe the unfolding technique.

\subsection{One-Dimensional Unfolding}
\label{ssec:afbunfolding1d}

Describe how we used 1D unfolding to measure the asymmetry variables.
Talk about binning choices, and all those other optimizations.

\subsection{Two-Dimensional Unfolding}
\label{ssec:afbunfolding2d}

Describe how we used 2D unfolding to measure differential asymmetries.
Talk about binning choices and other optimizations.

\subsection{Validation}
\label{ssec:afbunfoldingtests}

Talk about the linearity and pull tests, and how they quantify any bias
introduced by regularization.
Give the values from those tests.

\section{Systematic Uncertainties}
\label{sec:afbsystematics}

Describe all the systematic uncertainties we calculated.

\section{Results and Interpretation}
\label{sec:afbresults}

\subsection{One-Dimensional Results}
\label{ssec:afbresults1d}

Give the 1D results

\subsection{Two-Dimensional Results}
\label{ssec:afbresults2d}

Give the 2D results

\subsection{BSM Interpretation}
\label{ssec:afbresultsbsm}

Not sure if this section should be included or not. If I keep it:
Talk about the search for chromo-magnetic dipole moments.

