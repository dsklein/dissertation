\chapter{The Large Hadron Collider and the CMS Detector}
\label{chap:hardware}

In order to study the properties of the Standard Model particles, and
to search for physics beyond the Standard Model, we need a way to
produce particles other than the simple up and down quarks and
electrons that make up everyday matter. And once we have produced
particles, we need a way to detect their presence, identify them, and
measure their properties.

This chapter will describe the hardware used to perform
the analyses described in Chapters \ref{chap:afb} and \ref{chap:stop},
namely the Large Hadron Collider and the CMS Detector. In addition, it
will describe how the CMS Collaboration uses software to make sense of
the readings it has collected from the CMS detector. Because these
systems are immensely complex, this chapter will not attempt to be
totally comprehensive, but will focus on the elements that are most
important for understanding the science presented in the next two
chapters. % Review to see if I'm actually meeting this goal

\section{The Large Hadron Collider}
\label{sec:lhc}

In order to study heavy particles, we must first create them, since
they are hard (or even impossible) to find in nature.
Einstein's famous equation, $E = mc^2$, tells us that mass and energy
are equivalent to one another. Physicists employ this principle to create
heavier particles by taking lightweight particles, accelerating them to very high
speeds, and smashing them together. Particles moving at high speed
have high kinetic energy, and when those particle collide, that energy
can be converted into mass, creating heavier particles. Because it is
so often conducted at high energies, particle physics is sometimes
referred to as high energy physics, or HEP.

The Large Hadron Collider, or LHC, is currently the largest and most
powerful particle collider ever built. % cite this
It is currently the flagship experiment of CERN, the European Organization
for Nuclear Research, located in Geneva, Switzerland.
The LHC is a ring-shaped structure some 27 kilometers in circumference,
buried underground beneath the border between Switzerland and
France. The ring encloses two pipes, through which beams of
protons normally circulate in opposite directions. Strong electric fields
propel the protons along the beam pipes, and strong magnetic fields
steer the protons around the ring.

% Pictures of the LHC from above, and the tunnel.
% And maybe a cross-section of the machine.

The LHC is fed by a chain of smaller accelerators at CERN that shape
the beams and bring them partway up to speed so they can be injected
into the LHC. The LHC was designed to collide protons at an energy of
14 trillion electron-volts (TeV). However, during early testing, a
defective weld caused severe damage to the LHC; as a precaution
against future damage, the LHC was operated at 7 and 8 TeV energy in
its first run (Run I), and has been colliding at 13 TeV so far in its
second run (Run II). At these energies, the protons in the beam are
moving at 99.99999999\% of the speed of light.

The proton beams are not continuous streams of particles. Rather, the
protons are grouped into bunches. During Run I, the LHC circulated XX
bunches of YY particles. These bunches were spaced so that collisions
could be made to occur every 50ns, or 20 million collisions per
second. In Run II, the LHC circulates XX bunches of YY protons, and
collides them every 25ns (40 million collisions per second). % Bunch size and spacing needed!

Although we have been describing protons as simple particles, in fact
they are anything but. Protons contain two up quarks and a down quark,
and innumerable gluons binding them together. But in addition, protons
contain a sort of froth of quarks and antiquarks that are constantly
popping into and out of existence on ultrashort timescales, in
accordance with the laws of quantum mechanics. Collectively, all of
these components of a proton are known as \emph{partons}, and any of
them may be involved when two protons collide. Studies on the relative
momentum of these various partons tell us that at the LHC, most of the
time it is the gluons that are colliding with each other. % Citation?

At four places around the LHC ring, the beam pipes intersect, and the
proton beams are made to collide with one another. At each of these
four locations, a detector is placed where it can observe the
particles that are produced by these collisions. Two of
the detectors are large, general-purpose machines, designed to support
a wide variety of particle physics research. These detectors are called CMS
and ATLAS. There are two general-purpose detectors so that the results from one
may be cross-checked by the other. The other two main
detectors are moderately sized and designed to study specific kinds of physics
phenomena. One is LHCb, which is designed to study the physics of
bottom quarks. The other is ALICE; a few weeks each year, the LHC
collides lead nuclei instead of protons, and ALICE is designed to
study the physics of these collisions. In recent years, three smaller detectors have
been added: ToTeM, which shares an interaction point with CMS; MoEDAL,
which is colocated with LHCb; and LHCf, installed around the ATLAS detector.

\section{The CMS Detector}
\label{sec:cms}

The research described in Chapters \ref{chap:afb} and \ref{chap:stop}
was conducted using data gathered on proton-proton collisions by the
CMS detector. CMS stands for Compact Muon Solenoid, a name whose
origin will become apparent in the next few sections. In brief, though,
this detector is a complex, multilayered array of sensors designed to
detect and measure the properties of as many of the particles emerging
from the collision point as possible.

% Wedge diagram goes here

The CMS detector is composed of multiple layers, each with a specific
purpose. These layers work in concert to provide a fairly complete
picture of the numerous particles emerging from a collision.
Figure \ref{} shows a small cross-sectional wedge
of the detector, and demonstrates how each of the components
contributes to measurements of the particles produced in the
collisions.

The CMS detector was designed with a wide array of physics goals in
mind. Some of these goals included discovering the Higgs boson,
especially in decays to leptons or photons; discovering signs of
supersymmetry, if such signs existed; searching for extra
dimensions; and conducting precision tests of the predictions of the Standard Model.
These and other goals had to be balanced against practical
constraints, such as budget, durability, and ease of readout. The
detector that resulted from the convergence of these factors is
described in Sections
\ref{ssec:cms:components:magnet}-\ref{ssec:cms:components:muon}.

\subsection{Coordinates}
\label{ssec:cms:coordinates}

% Insert pictures that show detector coordinates

Before we can understand the design of the CMS detector, or the
physics it studies, we must assign a coordinate system to the detector.
Because the beam pipe has cylindrical symmetry, the CMS detector is
similarly cylindrical. It is shaped like a giant barrel wrapped around
the beam pipe, some 8 stories tall and wide, and 10 stories % Check numbers
long. Figure \ref{} depicts the coordinate system attached to the
detector. The z-axis runs down the center of the beam pipe, and
defines a longitudinal axis. It has a positive and negative direction;
the point z=0 is the nominal spot where the protons collide. The
azimuthal direction in the detector is measured using the coordinate
$\phi$. The radial coordinate $r$ expresses perpendicular distance from the z-axis.

Because the proton beams are compressed and stabilized in the
transverse plane before they enter the collision area, we expect
that the system of two colliding protons has no net momentum
transverse to the beam pipe. We can therefore expect the total
\emph{transverse momentum} (or $p_T$) of the collision products to be
zero. However, because the individual partons inside a proton have
constantly-changing momenta, we cannot know the total longitudinal
momentum ($p_z$) of the pp collision. For
this reason, we consider the $p_T$ of particles almost exclusively,
and seldom spare a thought for $p_z$.

Because particle collisions take place almost exclusively near the
origin of the detector and their components spread out from there, it
is sometimes helpful to describe the trajectories of particles in terms of
a sort of angle that they make in the $r-z$ plane. We could use the
polar angle $\theta$ from spherical coordinates. But in a world of
particles moving at relativistic speeds, we will often need to boost
between different reference frames, and $\theta$ becomes
cumbersome. Physicists came up with a more useful coordinate, which
they termed \emph{rapidity}. In collider physics, the rapidity, $y$,
of a particle is given by:
\begin{equation}
\label{eq:cms:rapidity}
y = \frac{1}{2} \ln \frac{E+p_z}{E-p_z}
\end{equation}
This coordinate has the advantage of transforming simply under Lorentz
boosts. If two reference frames differ by a relativistic velocity
$\beta = v/c$, then the rapidity in the new frame is given by $y\prime
= y - \tanh^-1 \beta$. And because of this easy transformation rule,
the rapidity difference $\Delta y$ between two particles is Lorentz
invariant--it remains the same no matter what reference frame it is
measured in.

Rapidity will be used to define other variables in Section
\ref{ssec:afb:variables}, but on the whole, it turns out to be not
very convenient, because it requires us to know the energy of the
particle. So physicists have devised a related quantity that behaves
somewhat better, calling it \emph{pseudorapidity}. The pseudorapidity,
$\eta$, of a moving particle is defined by
\begin{equation}
\label{eq:cms:eta}
\eta = -\ln \left[ \tan \left( \frac{\theta}{2} \right) \right]
\end{equation}
This coordinate depends only on the polar angle, not the energy, of a
particle. And like its cousin, the pseudorapidity difference $\Delta\eta$
between two particles is also a relativistic invariant. In fact, at
velocities approaching the speed of light, rapidity and pseudorapidity
become equal to one another. $\eta$ is used widely in collider
physics, even to the extent that parts of the CMS detector are
referred to by their ($\eta, \phi$) coordinates.

% Describe the barrel and endcap regions. Necessary for explaining the
% geometry of the components.


\subsection{Superconducting Solenoid}
\label{ssec:cms:components:magnet}

The defining feature of CMS, which gives rise to the 'S' in its name,
is the superconducting solenoid. This component is a large
electromagnet; it is made from superconducting niobium-titanium
material, taking the form of a giant coil 12.9m long and with an inner
bore of 5.9m \cite{tdr}. When cooled to about 4 Kelvin using liquid helium,
the magnet can generate a longitudinal magnetic
field of up to 4T inside the solenoid, though in practice we run it at
3.8T to help it last longer. The return field outside the
solenoid is lower, and is not uniform \cite{accelexper}.

The purpose of this magnet is to bend the trajectories of particles
passing through the detector. The tracker and the muon system
(described below) can measure the radius of curvature of a track, from
which we can determine how much momentum the particle was
carrying. The strength of the magnetic field was chosen based on
the requirement that the CMS detector be able to determine the charge
of a muon carrying 1 TeV of momentum \cite{tdr}.

\subsection{Inner Tracker}
\label{ssec:cms:components:tracker}

The innermost component of the detector, wrapped immediately around
the beam pipe, is the inner tracker. This component is made of several
layers of silicon sensors. Every time a charged
particle passes through a layer of silicon, it creates a blip of
electrical current in the silicon that can be read out
electronically. We can ``connect the dots'' to reconstruct the path
the particle took. Because of the magnetic field inside the detector,
charged particles will have curving trajectories. Using the tracker,
we can measure the radius of curvature of these tracks, and thus
determine the momentum of the particle that created the
tracks. Because the magnetic field is only in the z-direction, we can
only measure the $p_T$ of the charged tracks, not the
$p_z$. Importantly, neutral particles do not leave any hits in the
tracker, and thus their trajectories cannot be measured.

% Figure of the tracker geometry

Figure \ref{} shows the geometry of the tracker as originally designed.
The innermost part of the tracker ($r \lesssim$ 10cm), where the tracks are densest,
consists of silicon pixels. At installation, there were three pixel
layers in the barrel and two in the endcaps. Each pixel is
100$\times$150 $\mu$m in size. At this scale, the expect occupancy of
a pixel was $10^{-4}$ per bunch crossing \cite{tdr}. The pixel tracker
was upgraded during the 2016/2017 winter shutdown, adding a fourth
pixel layer in the barrel and a third in the endcaps, while slimming
down the hardware that supports the functioning of the pixels
\cite{pixeltdr}.

The remaining layers of the tracker consist of silicon strips. In the
region $20 < r < 55$ cm in the barrel (Tracker Inner Barrel, TIB),
there are four layers of silicon
microstrips, with a size of at least 10 cm $\times$ 80$\mu$m. In the
Tracker Outer Barrel (TOB), defined by $r >$ 55cm, there are six
layers of microstrips with a size of no more than 25 cm $\times$
180$\mu$m. In the endcap regions, the Tracker Inner Disks (TID) sit
just outside the TIB, and consist of threel layers of disks of
strips. The Tracker Endcaps (TEC) sit just outside the TOB, and
consists of nine disks of silicon strips. These various layers provide
tracker coverage out to $|\eta| \approx 2.5$.

The electrical impulses produced in the silicon pixels and strips are
read out by ``APV25'' chips that amplify and process the signals. The
signals are then passed out of the detector using optical cable, and
further processed in hardware outside the detector itself
\cite{tdr}. Additional hardware circulates a refrigerant liquid that
maintains the silicon sensors at a temperature no higher than
$-10^\circ$ C \cite{accelexper}. The materials and geometry of the
original tracker are chosen to minimize the amount of energy absorbed
from particles as they pass through. The original
tracker had a radiation thickness of about $0.4 X_0$ at $\eta = 0$,
increasing to a maximum of $1.0 X_0$ at $|\eta| = 1.6$ and dropping
back to $0.6 X_0$ at $|\eta|=2.5$, \cite{tdr}. The radiation length,
denoted $X_0$, will be explained in the ECAL discussion next.
% Consider pulling plot of material budget

The performance of the inner tracker has been measured using both
muons and pions. For muons with $p_T$ of 100 GeV, the resolution
in $p_T$ is about 1.5-2\% up to $|\eta|$ of 1.6, beyond which the
resolution worsens exponentially. For muons with $p_T$ of 10 or 1 GeV,
the resolution is better, on the order of 0.5-1.0\%. The efficiency of
global muon track reconstruction is generally around 99\% for $p_T$ of
1, 10, or 100 GeV, with a slight dropoff at $\eta=0$ and an
exponential faloff above $|\eta|$ of 2.0. For pions, the global track
resolution is somewhat less, ranging from 85-95\% for 100 GeV pions and
75-90\% for 1 GeV pions \cite{tdr}.
% Consider pulling some of the resolution plots

\subsection{Electromagnetic Calorimeter (ECAL)}
\label{ssec:cms:components:ecal}

Just outside the tracker, the next component in the CMS
detector is the electromagnetic calorimeter, or ECAL. The purpose of
this device is to measure the energy of electromagnetic particles
(electrons and photons) emitted from the collisions. The ECAL consists
of a giant array of lead tungstate (PbWO$_4$) crystals; when struck by
electromagnetic particles, these crystals scintillate with a
pale blue light that can be read out by optical sensors. The amount of
light gives us a measure of how much energy the particle was
carrying. In addition, the pattern of energy deposition in the
crystals can provide information useful in particle
reconstruction.

Lead tungstate has a number of properties that make it an ideal
material to use in the ECAL. For one thing, it is quick to read
out: lead tungstate crystals can release 80\% of their scintillation
photons in the 25ns window between proton bunch crossings. In
addition, lead tungstate is highly resistant (``hard'') to the high
levels of radiation emitted by the collisions, tolerating a total
absorbed dose of up to 10 Mrad (100 kGy). But perhaps most importantly,
lead tungstate allows the ECAL to be built relatively compactly,
because the material has a short radiation length ($X_0$) and
Moli\`{e}re radius ($R_M$).

Radiation length and Moli\`{e}re radius are two properties that
describe a material's ability to absorb electromagnetic energy. When
an electron passes into a material, the radiation length is the
characteristic distance over which that electron will lose all
but 1/$e$ of its energy. To put it mathematically:
\begin{equation}
\label{eq:cms:ecal:radlength}
E(x) = E_0 \cdot e^{-x/X_0}
\end{equation}
It is also equal to 7/9 of the mean free path for pair production by
photons \cite{pdg}. So the shorter the radiation length for a
material, the shorter the distance needed for electromagnetic
particles to be stopped and their energy absorbed. Lead tungstate has
a radiation length $X_0 = 0.89$ cm \cite{tdr}. Similarly, the
Moli\`{e}re radius governs the absorption of energy in the direction
transverse to the particle's trajectory. It is the characteristic
lateral distance over which 90\% of an EM shower's energy will be
contained. So materials with a shorter Moli\`{e}re radius will have on
average a smaller transverse shower size. For lead tungstate, $R_M =
2.2$ cm \cite{tdr}. These characteristic distances will play a role in
determining the size of the ECAL crystals.

The ECAL is divided into two regions: the barrel, and the
endcaps. These may be seen in Figure \ref{}.
The PbWO$_4$ crystals are shaped like tapered rectangular prisms. In
the barrel region, they are 230 mm long, corresponding to 25.8 $X_0$,
and have front faces that are 22$\times$22 mm, corresponding to
1$\times$1 $R_M$. These dimensions ensure that very little EM energy
will escape out the back of the crystals, and that
about 94\% of a given shower will be contained in a 3$\times$3 array
of crystals. In the endcaps, the crystals are 220 mm long (24.7 $X_0$),
and have front faces of 28.6$\times$28.6 mm (1.3$\times$1.3 $R_M$).
The barrel section is composed of 36 ``supermodules''. Each
supermodule is an array of 85$\times$20 crystals, covering half the
barrel length and 20$^\circ$ in $\phi$. Each of the two endcaps is
composed of two half-circle structures called ``Dees''. Each dee holds
138 groupings of 5$\times$5 crystals (called``supercrystals'') plus
18 partial supercrystals. The endcaps are also fronted with preshower
detectors, whose purpose is primarily to study neutral pions produced
at high $\eta$. These preshower detecters are composed of lead
absorbers that initiate pion showering, and silicon strip detectors
that measure the size and shape of the shower. All told, there are
75,848 crystals in the entire ECAL, providing very fine spatial
granularity. As Figure \ref{}
shows, ECAL coverage extends out to $|\eta|$ = 3.0. However, there is
a gap in coverage between the barrel and the endcaps. Electromagnetic
particles that fall into this ``crack'' will not be measured, a fact
that we must account for when attempting to reconstruct electrons and
photons \cite{tdr}.

% Pull TDR figure 4.1

Lead tungstate produces a blue-green scintillation light, peaking near
420mm. The amount of light produced in each crystal must be measured,
and the information digitized, in order to determine the energy of incident EM
particles. This presents a slight problem. Despite its other admirable
properties, lead tungstate produces relatively little scintillation
light - about 30 photons per MeV of energy. This fact necessitates
the use of systems that can amplify faint signals. In the
barrel, photons are read out by silicon avalanche photodiodes (APDs)
stuck to the backs of each crystal. In the endcaps, vacuum
phototriodes (VPTs) are used for readout instead. These sensors feed
their signals to amplification and digitization hardware attached to
the CMS detector, and from there are sent to computing equipment
outside the detector volume \cite{tdr}.

The performance of the ECAL was measured using a controlled beam of
electrons. The measured energy resolution in groups of 3$\times$3
crystals is shown in Figure \ref{}.
This resolution function was parameterized by a Gaussian fit of the form
\begin{equation}
\left( \frac{\sigma}{E} \right)^2 =
\left( \frac{S}{\sqrt{E}} \right)^2 +
\left( \frac{N}{E} \right)^2 + C^2
\end{equation}
where $S$, $N$, and $C$ are parameters representing the stochastic,
noise, and constant contributions to the resolution, respectively. As
Figure \ref{} shows, the energy resolution is better
than 1\% for electron energies above about 20 GeV \cite{tdr}.

% Stick in TDR figure 1.7

As they pass through the tracker, electrons tend to produce
\emph{bremsstrahlung}, a shower of photons that radiate off as the
electron is decelerated. These photons tend to then convert into
electron-positron pairs, which may themselves produce bremsstrahlung,
and so on. This shower, combined with the curved trajectory of
electrons, means that electrons often appear in the ECAL as an energy
deposit with a long tail in the $\phi$ direction. Photons, by
contrast, are not bent by the magnetic field, and thus tend to create
clustered energy deposits wherever they strike the ECAL. CMS
scientists have a number clever ways to analyze the shape of these
energy deposits to aid with particle identification. This information
can also be combined with information from the tracker to
differentiate electrons and photons.
% Not sure if this info belongs here, or down below with particle ID.

\subsection{Hadron Calorimeter (HCAL)}
\label{ssec:cms:components:hcal}

Outside the ECAL but (mostly) inside the magnet lies the hadron
calorimeter, or HCAL. This component is designed to measure the energy
of hadronic particles produced in collisions. The HCAL is composed of
alternating layers of metal, used to absorb some of the incident
hadronic energy, and scintillating materials, used to measure the
hadronic energy deposited in the calorimeter. The HCAL thus has an
important role in measuring the energy of jets, as well as in
measuring pileup from secondary collisions in the event. % Does HCAL play a role in measuring pileup?

The power of materials to stop relativistic particles is described in
terms of the interaction length, $\lambda_I$. When a number of
particles are incident on some material, this parameter describes
the length over which all but a 1/$e$ fraction of those particles will
be absorbed. In other words:
\begin{equation}
\label{eq:cms:hcal:intlength}
N(x) = N_0 \cdot e^{-x/\lambda_I}
\end{equation}
So the shorter a material's interaction length, the shorter the
distance over which it will absorb incident particles.

The CMS HCAL is a \emph{sampling} calorimeter, meaning that it uses
different materials to produce the shower and to measure the
energy. Hadronic showering is induced by the absorber layers, most of
which are made of brass (70/30\% Cu/Zn). Brass
is used because it is relatively affordable, and also because it is
non-magnetic, and thus won't perturb the magnetic field inside the
detector. This particular brass alloy has an interaction length
$\lambda_I = 16.42$ cm. Stainless steel is also used as an absorber in places.
Interleaved with the absorber materials are layers of
scintillating material. Most of the scintillators are made of
radiation-hard plastic, either Kuraray SCSN81 or Bicron BC408, though
where extreme particle flux and radiation is an issue, quartz fiber is
used instead for its superior radiation hardness.

The HCAL is divided into four subcomponents: the barrel (HB), the
endcaps (HE), the outer calorimeter (HO), and the forward calorimeter
(HF). Each of these components operates on the same principles, though
the design of each differs slightly. These components are all cleverly
overlapped to avoid any cracks like that of the ECAL.

The barrel covers the range $|\eta| <$ 1.3. The absorber material in
the HB is segmented into 36 wedges; each wedge encompasses half the
length of the HB and 20$^\circ$ in $\phi$. There are 16 layers of
absorber material in each wedge; the innermost and outermost layers are
made of stainless steel, for strength, and the middle 14 layers are
brass. The total thickness of these absorber layers is 5.82
$\lambda_I$ at $\eta$ = 0, increasing to 10.6 $\lambda_I$ at $|\eta|$
= 1.3. Within each wedge, the scintillating material is further
divided into segments of size 0.087$\times$0.087 in $(\Delta\eta,\Delta\phi)$
space, providing fine granularity. The innermost layer of
scintillating material in the HB is Bicron BC408, and the remaining 16
layers are Kuraray SCSN81 \cite{accelexper}.

The endcaps cover the regions 1.3 $< |\eta| <$ 3.0, and use the same
absorber and scintillator materials as the barrel, with one exception:
stainless steel is only used as the outer absorber layer, to
prevent any magnetic interference inside the magnet bore. The
scintillators are divided into segments of size 0.087$\times$0.087 in
$(\Delta\eta,\Delta\phi)$ space for $|\eta| <$ 1.6, and approximately
0.17$\times$0.17 for $|\eta| \geq$ 1.6. The total thickness of the
endcaps (including attached ECAL dees) is about 10 $\lambda_I$
\cite{accelexper}.

The outer calorimeter is designed to augment the stopping power of the
HB and the EB, and covers the region $|\eta| <$ 1.3. It consists of
tiles of Bicron BC408 scintillator embedded in the iron yoke that
gathers the returning magnetic field outside the solenoid. Thus the HO
actually uses the solenoid material itself as an absorber.
Following the shape of the return yoke, the HO is divided into 5 rings
along the $z$-axis, each of which has 12 sectors in $\phi$. There are
gaps between the rings and in some azimuth sectors for the cryogenic
and power lines that supply the magnet. The scintillator tiles roughly
follow the 0.087$\times$0.087 segmentation of the HB, within the
constraints of the yoke geometry \cite{accelexper}.

The forward calorimeter is located in the region 3.0 $< |\eta| <$
5.2. This area receives an extremely high flux of particles due to its
small angle with respect to the beamline. As such, this component
must be considerably more radiation hard than any other part of the
HCAL. To meet this requirement, the HF uses quartz fibers with polymer
cladding as its measuring material, and reads out Cherenkov light rather than
scintillation light. Each end of the HF is composed of stainless steel
absorbers arranged in 18 azimuthal wedges, each of which is penetrated
by quartz fibers that run parallel to the beamline. Some of these
fibers only penetrate partway through the steel plate, allowing the HF to
differentiate between electromagnetic and hadronic showers based on
penetration depth. The fibers form towers of size 0.175$\times$0.175
in $(\Delta\eta,\Delta\phi)$ space \cite{accelexper}.

The scintillation light from the plastic tiles is read out by
Kuraray Y-11 wavelength-shifting (WLS) fibers. These fibers are embedded
into the tiles themselves. Once outside the tiles, the WLS fibers are
spliced to clear optical fibers that send the light to hybrid
photodiodes for electronic processing. The Cherenkov light from the
quartz fibers is transmitted to air-core light guides, which carry the
light through layers of radiation shielding to photomultiplier tubes
outside the shielding \cite{accelexper}.

% Resolution/performance?
   % Figure 1.8 apparently shows that JER as a function of ET is similar in all parts of the HCAL
   % MET resolution function is given in 1.5.4.5
   % May need to see TDR chapter 11? (This according to 1.5.4.5).
   % See also TDR 5.4?

\subsection{Muon System}
\label{ssec:cms:components:muon}

The outermost component of the CMS detector is the muon system. The
strong penetrating power of muons allows us to place this system
outside all the other layers without fear that the muons will be
absorbed en route. As its name suggests, the muon system is
responsible for measuring the momentum of muons as they fly away from
the collision point. It employs three different gas-and-electrode
technologies to reconstruct the trajectories of muons, from which
momentum can be inferred. These trajectories can be combined with
measurements from the tracker for greater precision.

% Insert Fig. 1.6 from the TDR

The layout of the muon system is presented in Figure \ref{}.
Like many other components, it is divided into barrel and endcap
regions. The barrel region detects muons using a combination of drift
tubes (DTs) and resistive plate chambers (RPCs), whereas the endcaps use cathode
strip chambers (CSCs) and RPCs. All three of these technologies detect
muons by the trail of ionization they leave after passing through a gas. The
liberated electrons will be attracted to positively charged
electrodes, and the ions to negatively charged electrodes. When
electrons and ions hit the electrodes, they produce an electrical
signal that is read out, and timing information
is used to tell how far along the electrode the impact occurred.
The drift tubes are long, thin chambers filled with a mixture of 85\%
Ar and 15\% CO$_2$ gases, and with a positively charged wire running
down their centers. Drift times in these chambers are a maximum of 380
ns. These tubes provide 1D position measurements, but multiple layers
may be stacked at right angles to
provide 2D measurements. These devices are used because they are precise
and inexpensive, and can function well in the muon barrel, where the
particle flux and magnetic field are both low
\cite{accelexper,websitedt}. The cathode strip chambers are planar
chambers filled with a mix of 40\% AR, 50\% CO$_2$, and 10\%
CF$_4$. They contain positively charged wires running in
one direction, and negatively charged strips running perpendicularly,
providing native 2D position measurements. CSCs are employed in the
endcap because they are precise, moderately fast ($<$ 225 ns), and can operate
in the high magnetic fields at the fringes of the solenoid
\cite{accelexper,websitecsc}. Resistive plate chambers are planar
chambers filled with a mixture of 96.2\% C$_2$H$_2$F$_4$, 3.5\%
$i$C$_4$H$_{10}$, and 0.3\% SF$_6$. They have a positively charged
plate on one face, and a negatively charged plate on the
opposite face. Electrons are actually detected by metal strips just outside
the chambers. RPCs are used to supplement the DTs and CSCs
mainly for their 1 ns timing resolution; their spatial
resolution is not as fine as the DTs or the CSCs
\cite{accelexper,websiterpc}. The signals from these subsystems are
processed in electronics both within the detector volume and outside it.

The barrel section of the muon system, covering $|\eta| <$ 1.2, is
interleaved with the iron return yokes, structures that concentrate
the magnetic field exiting the solenoid and return it around to the
other end. As such, the muon barrel follows the geometry of the
yokes. The return yokes consist of five rings % width of the rings?
and divided into 12 sectors in $\phi$. As Figure \ref{} shows,
the DTs and RPCs are stacked in four layers, or \emph{stations}. These
are offset in $\phi$ so that all muons will pass through at least
three. The three innermost stations contain 12 planes of DTs, 8 of
which provide $r-\phi$ measurements, and 4 of which provide $z$
measurements. The outer station lacks the $z$ measuring planes. The
inner two stations are coupled to two RPCs each, and the outer two
stations have one RPC each. With this geometry, each individual
station can provide position measurements with a precision of better
than 100 $\mu$m in space and about 1 mrad in $\phi$ \cite{tdr}.

The endcap muon system covers the range 0.9 $< |\eta| <$ 2.4. The
CSCs are arrayed in four layers of disks, with each disk being made up
of several trapezoidal chambers 10 or 20$^\circ$ wide in $\phi$,
arranged in concentric rings. Each chamber contains 6 gas gaps and
electrode grids. There are 36 chambers in each ring, except the
centermost rings of stations 2-4, which have only 18 chambers. The
CSCs provide a spatial resolution of about 200 $\mu$m and an angular
resolution in $\phi$ of about 10 mrad \cite{tdr}. During Run I, the
first three disks had RPCs attached to all but the central ring; for
Run II, the outer CSC ring of disk 4 was added, and was instrumented
with attached RPCs.

% Anything I want to say about cosmics?

\section{Collecting and Interpreting the Data}
\label{sec:cms:datacollectinterp}

\subsection{Triggers} % and other stuff?
\label{ssec:cms:triggers}

% We can't keep all the data we capture.
% We use hardware and software triggers to pick out the events that are
% most interesting, or clearest.
%
% Some of this hardware is embedded directly in the detector,
% e.g. muon systems can directly trigger L1
%
% Later on down the line, we sometimes have to correct for trigger effects.
% Sometimes we have to prescale the triggers.
% Triggers are used to sort the events into primary datasets.

\subsection{Particle reconstruction and identification}
\label{ssec:cms:recoandid}

% We use the particle flow algorithm to make candidates out of the readouts.
% Candidates may or may not actually be the thing they're ID'ed as.
% So, we have IDs with different working points, which vary from analysis to analysis.
% MET is also a thing.

\subsection{Monte Carlo simulations}
\label{ssec:cms:montecarlo}

% The easiest way to compare theory predictions against our data is to
% make fake data based on the theory equations.
% There are a bunch of software packages we use to do this.
% Some of these packages include MadGraph, AMC@NLO, Powheg, Pythia, etc.
% We also use GEANT to simulate the detector response to particles.
% (OSCAR is based on GEANT)
% At the end of the day, we have tons of fake data we can use for various purposes.

\section{Acknowledgements}
\label{sec:hardware:acknowledgements}

