\chapter{Single-Lepton Stop Search}
\label{chap:stop}

This chapter will describe a search for supersymmetry with the target signal
being stop-antistop squark pairs decaying to a single-lepton (1$\ell$)
final state. This work was performed using the CMS experiment during
Run II of the LHC, at 13 TeV center-of-mass energy. This analysis
resulted in two publications: the search performed using 2016 data \cite{stop1l},
and a combined single-lepton and all-hadronic search using 2015 data
\cite{combination0l}. Two public research documents (PASes) were
also produced to support conference results \cite{pasichep,pasmoriond},
however these are superceded by the published results. I will focus on
the analysis as described in Reference \cite{stop1l}, particularly my work
developing the compressed T2tt search strategy.

\section{Motivation}
\label{sec:stop:motivation}

As Section \ref{ssec:susy:rationale} has described, supersymmetry
(SUSY) is a very important class of theories in particle physics. It
has the potential to solve the hierarchy problem, and may provide the
answer to the difficult question of what particles make up dark
matter. And the notion of naturalness would seem to imply that our
current or near-future particle colliders have just the right energies
to search for evidence of SUSY.

One particularly interesting feature of SUSY is that if you arrange
the sparticles in order from lightest to heaviest, they will generally line up
in reverse order from the Standard Model particles. So whereas the
electron is the lightest stable SM particle, and the top quark is the
heaviest, by contrast the stop squark is expected to be one of the lightest
sparticles, and the selectron one of the heaviest. We say SUSY has an
\emph{inverted mass hierarchy}. % Cite this!
% See refs 8-12 in 8 TeV stop paper.
This means that the Lightest Supersymmetric Particle ($\lsp$), the
chargino ($\chargino$), and the stop squark ($\tilde{t}$) should be
some of the easiest sparticles to detect.
% The single-lepton channel provides a nice balance between
% cross-section and ease of ID.

\section{Previous Searches}
\label{sec:stop:run1}

A search for stop pairs in the one-lepton final state was previously
performed at CMS during Run I \cite{stop1l8tev}. This search used 19.5
fb$^{-1}$ of 8 TeV collision data. The analyzers performed a traditional
cut-based search, and also a search employing a boosted decision tree
(BDT). This machine-learning technique allows a computer to attempt to
discriminate between signal and background. Ultimately, neither search
strategy detected any evidence of the production of stop squarks. This
result allowed the analyzers to set limits on the possible masses of
stops and LSPs. Specifically, stop squarks were excluded up to masses
of about 650 GeV in the case where the LSP is massless, and LSPs were
excluded up to a maximum of 200-250 GeV for stop masses around 500-600
GeV.

The LHC and the CMS detector received a number of upgrades for Run II,
some of which have greatly benefitted the single-lepton stop
search. Of particular note, the LHC collision energy was raised from 8
to 13 TeV, which increases the likelihood of producing heavy new
particles, and the luminosity was increased considerably, allowing us % Quantify lumi increase
to record more data in the same amount of running time. In addition,
we have learned from the Run I analysis, and attempted to improve our
analysis techniques for Run II. To avoid the obfuscation inherent in
results produced by machine learning, we declined to perform a BDT
search. But we did add a dedicated search strategy to address the
previous lack of sensitivity in the ``top corridor'' region, as will
be discussed in Sections \ref{ssec:stop:sigcompressed} and
\ref{sec:stop:sigregs}.

% How about any ATLAS stop-1L searches?
% Or searches from CDF or D0?
% It's hard, because there are a LOT of searches that set limits on
% stops in general.

\section{Signal Models}
\label{sec:stop:sigmodels}

\subsection{Bulk Signals}
\label{ssec:stop:sigbulk}

Talk about T2tt production in the bulk region.
Maybe also describe other signal models?
E.g.~T2tb?

\subsection{Compressed T2tt}
\label{ssec:stop:sigcompressed}

Talk about compressed T2tt production, and why it's different from bulk T2tt.
In particular, discuss how those differences play out in the decay products.

\section{Backgrounds}
\label{sec:stop:bkgs}

Discuss the major background processes, and what's done to
reduce their prevalence in the data.
Unsure whether to group by production mode or final state.

\section{Datasets and Triggers}
\label{sec:stop:datatrig}

Go over which datasets were used, and which triggers.

\section{Object and Event Selection}
\label{sec:stop:selections}

Give the criteria we used to define electrons, muons, hadronic taus,
jets, b-tags, MET, etc.
Talk about how we used these objects to select events.
Maybe this is a good place to also describe the scale factors and
other corrections applied to the Monte Carlo.

\section{Signal Regions}
\label{sec:stop:sigregs}

Give the definitions for the different signal regions.
Talk about why these definitions were chosen.

\section{Background Estimation}
\label{sec:stop:bkgest}

If you didn't do so earlier, go over the different backgrounds:
Lost lepton, 1l-from-W, 1l-from-top, and rare.

\subsection{Lost Lepton}
\label{ssec:stop:lostlep}

Introduce the dilepton control regions, and the selections used
to define them.
Talk about how these control regions are validated.
Describe how we do a data-driven estimate based on the CR yields.
Explain how we estimate all the various systematic uncertainties.

\subsection{Single Lepton not from Top}
\label{ssec:stop:1lw}

Talk about the single-lepton-from-W background.
Describe the WJets control regions and their selections.
Talk about validating these control regions.
Describe the data-driven estimate for this component.
Explain how the systematic uncertainties are calculated.

\subsection{Single Lepton from Top}
\label{ssec:stop:1ltop}

Explain how we take this background from Monte Carlo.
Describe the uncertainties we use to cover this estimate.

\subsection{Rare Standard Model Processes}
\label{ssec:stop:1lrare}

Describe the rare background (particularly $TTZ \rightarrow \nu\nu$).
Talk about how this background is estimated.
Describe how systematics are assessed.

\section{Signal Estimate}
\label{sec:stop:signal}

Talk about how signal yields are estimated.
Describe the corrections for signal contamination in control regions.
Explain the method for assessing uncertainties on signal yields.

\section{Results}
\label{sec:stop:results}

Give the final yields and uncertainties.

\section{Limit Setting}
\label{sec:stop:limits}

Describe the Higgs Combine tool.
Talk about how we use it.
Explain (the rudiments of) the statistical methods.

\section{Interpretation}
\label{sec:stop:interp}

Maybe combine with previous section?
Anyway, interpret our results in the context of constraints on stop production.

