\chapter{Single-Lepton Stop Search}
\label{chap:stop}

\section{Motivation}
\label{sec:stop:motivation}

SUSY is a really important theory, for reasons described in section \ref{ssec:susy:rationale}.
Since the top is the heaviest particle, the stop is probably the lightest sparticle.
The single-lepton channel provides a nice balance between cross-section and ease of ID.

\section{Previous Searches}
\label{sec:stop:run1}

In Run I, CMS performed a single-lepton stop search.
Present a sketch of their results.
Maybe talk about what we do better in Run II?

\section{Signal Models}
\label{sec:stop:sigmodels}

\subsection{Bulk Signals}
\label{ssec:stop:sigbulk}

Talk about T2tt production in the bulk region.
Maybe also describe other signal models?
E.g.~T2tb?

\subsection{Compressed T2tt}
\label{ssec:stop:sigcompressed}

Talk about compressed T2tt production, and why it's different from bulk T2tt.
In particular, discuss how those differences play out in the decay products.

\section{Backgrounds}
\label{sec:stop:bkgs}

Discuss the major background processes, and what's done to
reduce their prevalence in the data.
Unsure whether to group by production mode or final state.

\section{Datasets and Triggers}
\label{sec:stop:datatrig}

Go over which datasets were used, and which triggers.

\section{Object and Event Selection}
\label{sec:stop:selections}

Give the criteria we used to define electrons, muons, hadronic taus,
jets, b-tags, MET, etc.
Talk about how we used these objects to select events.
Maybe this is a good place to also describe the scale factors and
other corrections applied to the Monte Carlo.

\section{Signal Regions}
\label{sec:stop:sigregs}

Give the definitions for the different signal regions.
Talk about why these definitions were chosen.

\section{Background Estimation}
\label{sec:stop:bkgest}

If you didn't do so earlier, go over the different backgrounds:
Lost lepton, 1l-from-W, 1l-from-top, and rare.

\subsection{Lost Lepton}
\label{ssec:stop:lostlep}

Introduce the dilepton control regions, and the selections used
to define them.
Talk about how these control regions are validated.
Describe how we do a data-driven estimate based on the CR yields.
Explain how we estimate all the various systematic uncertainties.

\subsection{Single Lepton not from Top}
\label{ssec:stop:1lw}

Talk about the single-lepton-from-W background.
Describe the WJets control regions and their selections.
Talk about validating these control regions.
Describe the data-driven estimate for this component.
Explain how the systematic uncertainties are calculated.

\subsection{Single Lepton from Top}
\label{ssec:stop:1ltop}

Explain how we take this background from Monte Carlo.
Describe the uncertainties we use to cover this estimate.

\subsection{Rare Standard Model Processes}
\label{ssec:stop:1lrare}

Describe the rare background (particularly $TTZ \rightarrow \nu\nu$).
Talk about how this background is estimated.
Describe how systematics are assessed.

\section{Signal Estimate}
\label{sec:stop:signal}

Talk about how signal yields are estimated.
Describe the corrections for signal contamination in control regions.
Explain the method for assessing uncertainties on signal yields.

\section{Results}
\label{sec:stop:results}

Give the final yields and uncertainties.

\section{Limit Setting}
\label{sec:stop:limits}

Describe the Higgs Combine tool.
Talk about how we use it.
Explain (the rudiments of) the statistical methods.

\section{Interpretation}
\label{sec:stop:interp}

Maybe combine with previous section?
Anyway, interpret our results in the context of constraints on stop production.

